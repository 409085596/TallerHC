\documentclass[letterpaper, 12pt, oneside]{article} %Para dar formato al documento
\usepackage{amsmath}
\usepackage{graphicx}
\usepackage{xcolor}
\usepackage[utf8]{inputenc}

\title{\Huge Bitácora 3 del Taller de Herramientas Computacionales}
\author{Elías Jiménez Cruz, 409085596}
\date{12/01/2019}

\begin{document}
	\maketitle
	\paragraph{\\Se checaron una vez más los pasos para crear y sincronizar un repositorio de git. Como ejercicio se crearon en la carpeta TallerHC la carpeta Clases, que contendrá los archivos creados durante el curso y que estará dividida en dos carpetas más: Latex y Programas. La primera carpeta contendrá nuestras bitácoras de cada día y los archivos de Latex que creemos, mientras que programas contendrá los documentos para Python que creemos por igual. Para esta creación se hizo uso del comando "mkdir" en Bash, donde se hizo la observación que si se escribía "mkdir -p "ruta de directorios" se podía crear más de una carpeta por comando, estableciendo una ruta parental. A continuación se creó un archivo de nombre Clase03.txt que se guardó en la carpeta de Latex, en el cual se anotaron los pasos para crear un repositorio en git. Para hacer este archivo se utilizó el comando vi en Bash, el cual lanza la aplicación Vim, un editor de texto plano que se encuentra dentro del mismo Shell. El comando a ejecutar desde la carpeta de Latex es "vi Clase03.txt" el cual abre Vim para la edición del texto. Mostrará el archivo vacío y para comenzar a editarlo es necesario teclear la letra "i". Se escribe el documento y una vez finalizada la edición se pulsa la tecla "esc" para dejar de editarlo. Ahora, para guardar y salir, se debe escribir el doble punto ":" seguido de "wq" (sin comillas) para entonces pulsar la tecla enter. El sistema regresará a Bash, y se podrá comprobar que el documento creado existe en la carpeta actual. Se puede usar el comando "cat Clase03.txt" para checar de forma sucinta lo escrito. Ahora bien, si se desea salir sin guardar, se debe escribir ":q!", con lo cual el sistema no guardará los cambios y, en su caso, no creará el archivo. A continuación, se debió subir los directorios y el nuevo archivo al repositorio de git, para ello se usaron los comandos "git add *", "git commit" y "git push", tal y como se describió en la bitácora anterior.\\Después de esta práctica, se procedió a }
\end{document}