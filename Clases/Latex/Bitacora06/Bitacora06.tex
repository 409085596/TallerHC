\documentclass[letterpaper, 12pt, oneside]{article} %Para dar formato al documento
\usepackage{amsmath}
\usepackage{graphicx}
\usepackage{xcolor}
\usepackage[utf8]{inputenc}

\title{\Huge Bitácora 6 del Taller de Herramientas Computacionales}
\author{Elías Jiménez Cruz, 409085596}
\date{14/01/2019}

\begin{document}
	\maketitle
	\paragraph{En python se vio el comando if, el cual permite que el sistema ejecute una instrucción con determinadas condiciones. Básicamente la estructura es if (condición): y debajo se escribe la instrucción con identación. Como ejemplos calculamos el valor absoluto de un número, en el que si era negativo lo devolvía positivo y si era positivo lo devolvía igual. Como ejercicio se dejó hacer diana.py y como tarea una extensión del mismo. En Latex vimos como poner secciones y símbolos matemáticos.}
\end{document}