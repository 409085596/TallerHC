\documentclass[letterpaper, 12pt, oneside]{article} %Para dar formato al documento
\usepackage{amsmath}
\usepackage{graphicx}
\usepackage{xcolor}
\usepackage[utf8]{inputenc}

\title{\Huge Bitácora 1 del Taller de Herramientas Computacionales}
\author{Elías Jiménez Cruz, 409085596}
\date{12/01/2019}

\begin{document}
	\maketitle
	\paragraph{Este es un compendio de preguntas de las bitácoras 1 a 5, a manera de guía de estudio.\\\\}
	\begin{enumerate}
		\item ¿Qué es un sistema operativo?\\R: Es un sistema que administra los recursos de una computadora y que permite la interacción entre ésta y el usuario.\\
		\item ¿Qué es un sistema operativo de código libre?\\R: Es un sistema operativo en el que cualquier persona, sin un estricto ánimo de lucro, puede tener libre acceso al código del sistema para así mejorarlo y colaborar en su desarrollo. Un gran ejemplo de estos sistemas es el sistema operativo Linux, creado por Linus Torvalds.\\
		\item ¿Qué es una distribución de Linux?\\R: Una versión del sistema, que tiene el kernel de Linux y características de rendimiento, eficiencia o versatilidad que le hacen diferenciarse del resto de las versiones.\\
		\item Menciona las tres distribuciones prinicipales de Linux.\\R: Ubuntu, Debian y Fedora.\\
		\item ¿Qué es un Shell?\\R: Es un intérprete de comandos, un programa que está a la espera de una instrucción, o comando, para ejecutarla. En Linux, el Shell se le conoce como Bash.\\
		\item ¿Para qué se utilizan los comandos de Bash?\\R: Para administrar y manejar el sistema, así como para navegar por sus directorios y visualizar y ejecutar sus archivos.\\
		\item Menciona tres comandos de Bash y explica su función.\\R: set, muestra las variables de entorno del sistema; top, muestra información sobre los procesos en ejecución y el procesador, básicamente es la versión del administrador de tareas de Linux; cd, que se utiliza para cambiar de ubicación en el árbol de directorios del sistema.\\
		\item ¿Para qué se utiliza el comando chmod? ¿Cómo se utiliza?\\R: El comando "chmod xxx "archivo"" cambia los permisos de lectura, escritura y ejecución de un archivo a nivel usuario, grupo y general. En cada x se modificarán los permisos respectivamente para los niveles mencionados, y se modificarán en base al siguiente criterio: Para activar la lectura del archivo se sumará cuatro a la casilla del nivel deseado, si se desea activar la escritura, se sumará dos y si se quiere activar la ejecución se sumará uno, dando un máximo de siete y un mínimo de cero para cada nivel.\\
		\item ¿Qué es git?\\R: Un servidor que permite guardar repositorios de proyectos de desarrollador.\\
		\item ¿Qué es Vim?\\R: Un editor de texto que puede ser utilizado desde Bash.\\
		\item ¿Cuáles son los pasos para solucionar un problema?\\R: Definir o entender el problema, analizar y delimitar el problema, buscar posibles soluciones, describir las soluciones con detalle, llegar a una solución general.\\
		\item 
	\end{enumerate}
\end{document}