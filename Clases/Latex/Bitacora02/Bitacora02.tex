\documentclass[letterpaper, 12pt, oneside]{article} %Para dar formato al documento
\usepackage{amsmath}
\usepackage{graphicx}
\usepackage{xcolor}
\usepackage[utf8]{inputenc}

\title{\Huge Bitácora 2 del Taller de Herramientas Computacionales}
\author{Elías Jiménez Cruz, 409085596}
\date{12/01/2019}

\begin{document}
	\maketitle
	\paragraph{\\En la clase 2 finalmente dejamos las charlas y en el taller de ciencias de la computación 1 comenzamos a trabajar en el entorno de Linux. Los que trajimos nuestro propio computador, instalamos el sistema operativo y ejecutamos los comandos "apt-get update" y "apt-get upgrade" para actualizar la biblioteca de donde se descargan aplicaciones. A continuación debimos realizar un paso importante para la dinámica del curso, que consistió en crear un repositorio público en el servidor git con el objeto de subir nuestros trabajos para su revisión. El procedimiento es sencillo en realidad, primero hay que crear la cuenta en github.com, con nuestro número de cuenta como nombre de usuario y nuestro correo de ciencias como el correo de la cuenta. Una vez creada la cuenta, se tiene que acceder a nuestros repositorios en github.com y entonces hacer clic en nuevo. Entonces se pedirá la confirmación de nuestro correo, con lo que se nos enviará un mensaje al correo para así verificarlo. Una vez hecho esto, se nos permitirá crear el repositorio, sólo habrá que nombrarlo. En mi caso particular, lo nombré como TallerHC. Ahora viene la parte más complicada, que es sincronizar ese repositorio con el Linux de nuestro computador. En Bash, hay que escribir el comando "apt-get install git", que instalará la aplicación de git. A continuación, hay que escribir "git init", que iniciará la aplicación, para entonces darnos de alta en la aplicación con nuestro correo y nombre de usuario, para lo cual escribimos los comandos "git config --global user.email "correo"" y "git config --global user.name "NombreDeUsuario"". Una vez registrada la cuenta que queremos sincronizar, tenemos que clonar el repositorio de git que ya creamos en el servidor (TallerHC), en la carpeta de nuestra preferencia. Para ello, vamos a github.com y nos dirigimos al repositorio recién creado, donde encontraremos una opción que dice "clonar o descargar". Damos clic ahí y copiamos la url que se nos presenta. Ahora nos dirigimos en Bash a la carpeta deseada y escribimos el comando "git clone "DirecciónQueCopiamos"" con lo que el repositorio completo se descargará a la carpeta deseada. Con esto estaremos preparados para empezar a guardar información en la carpeta descargada y así sincronizarla con el repositorio en git. Una vez se hayan guardado archivos en la carpeta TallerHC, para subirlos al servidor hay que ir en Bash a la carpeta de nuestro repositorio y escrbir "git add *" para que los archivos nuevos que no se hayan sincronizado previamente se preparen para la subida. Entonces hay que hacer un comentario de lo que vamos a subir, para ello escribimos el comando "git commit", donde se nos presentará una sección para escribir la reseña. En el caso de Slax, se presiona F2 para guardar el comentario y F10 para salir de la sección. Una vez hecho esto, se procederá a subir los nuevos archivos, con el comando "git push". El sistema indicará si se realizó bien el proceso o no. Ahora bien, se puede checar previamente si existen archivos pendientes de sincronizarse con el servidor, para lo cual se utilizará el comando "git status" el cual indica si hay archivos que no han sido sincronizados y si estos ya están en el proceso de sincronizarse. En el caso de que no haya, simplemente indicará que el árbol del directorio se encuentra limpio. Por otra parte, es posible que lo que se quiera hacer es descargar los archivos del servidor para actualizar el directorio que tenemos en nuestra computadora. Para ello, se puede usar "git status" para comprobar si existen archivos por descargar y en el caso afirmativo usamos "git pull" para la sincronización de descarga.\\De esta manera, en esta clase nos familiarizamos más en el entorno de Linux, y si bien hubo problemas para seguir los pasos en la ejecución de las actividades, el simple hecho de seguirlos fue de mucha utilidad para dejar de tener como extraño a un sistema que es usado por muchísimos desarrolladores por su característica intrínseca de seguridad, versatilidad y apertura de la información.}
\end{document}