\documentclass{beamer}
\usepackage{graphicx}
\usepackage[utf8]{inputenc}
%\usepackage[spanish]{babel}
\graphicspath{{../Imagenes/}}
%\usetheme{Antibes}
%\usetheme{AnnArbor}
%\usetheme{Berkeley}
%\usetheme{CambridgeUS}
%\usetheme{Goettingen}
%\usetheme{Hannover}
%\usetheme{Ilmenau}
%\usetheme{Berlin}
%\usetheme{Boadilla}
%\usetheme{Darmstadt}
\usetheme{Bergen}

\def\insertauthorindicator{¿Quién?}
\def\insertdateindicator{Fecha}
\title{Taller de Herramientas Computacionales}
\author{Elías Jiménez Cruz}
\date{\today}

\begin{document}
\maketitle
\begin{frame}
%\transboxin
\transblindshorizontal
\frametitle{Mi primera presentación en LaTeX}
\begin{center}
\includegraphics[scale=0.40]{EscudoFC.png}
\end{center}
\end{frame}
\begin{frame}
\frametitle{Segunda diapositiva}
Esta es mi segunda diapositiva
\end{frame}
\begin{frame}[fragile]
\begin{verbatim}
#!/usr/bin/python2.7
# -*- coding: utf-8 -*-

'''Elías Jiménez Cruz
409085596
Taller de herramientas computacionales
{aquí va una descripción del programa y lo que hace}'''

x= 10.5
y= 1.0/3
z= 15.3
#x,y,z= 10.5,1.0/3,15.3
H= '''El punto en R3 es (x,y,z) = (%.2f,%g,%G)''' %(x,y,z)
print H

G= '''El punto en R3 es (x,y,z) = ({laX:.2f},{laY:g},{laZ:G})''' .format(laX=x,laY=y,laZ=z)
print G

#import math as m
#from math import sqrt
#from math import sqrt as s

from math import *
x = input("Dame el número del cual quieres conocer su raíz: ")
print "La raíz cuadrada de %.2f es %f" %(x,sqrt(x))
\end{verbatim}
\end{frame}
\end{document}