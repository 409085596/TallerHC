\documentclass[letterpaper, 12pt, oneside]{article} %Para dar formato al documento
\usepackage{amsmath}
\usepackage{graphicx}
\usepackage{xcolor}
\usepackage[utf8]{inputenc}

\title{\Huge Preguntas 2 del Taller de Herramientas Computacionales}
\author{Elías Jiménez Cruz, 409085596}
\date{12/01/2019}

\begin{document}
	\maketitle
	\paragraph{Este es un compendio de preguntas de las bitácoras 6 a 10, a manera de guía de estudio.\\\\}
	\begin{enumerate}
		\item ¿Para qué sirve el comando if?\\R: Permite que el sistema se ejecute una instrucción si se ejecuta determinada condición.
		\item ¿Cómo sería el formato de if en Python?\\R: Básicamente la estructura es if (condición): y debajo se escribe la instrucción con identación.
		\item ¿Cómo se escribiría una fracción en Latex?\\R: \begin{verbatim} $\frac{numerador}{denominador}$ \end{verbatim}
		\item ¿Para qué sirve el comando while?\\R: Permite que una instrucción se repita cíclicamente hasta que se deje de cumplirse una función determinada. Básicamente se escribe while(condición): y debajo se pone de forma indentada la instrucción.
		\item ¿Qué es un valor booleano?\\R: Aquél cuyo valor es verdadero o falso.
		\item ¿Cómo se puede utilizar una biblioteca en Python?\\R: Antes de utilizarla se debe escribir el comando from "biblioteca" import "funciones" para poder usar dichas funciones en el programa en particular. Ahora bien, también se puede escribir import "biblioteca" pero para usar la función en esta manera se debe anteponer el nombre de la biblioteca más un punto.
		\item ¿Qué es una lista en Python?\\R: Una estructura de datos bastante importante para este lenguaje de programación. Si ha de definirse en lenguaje coloquial de alguna manera, es un conjunto ordenado, el cual consta de elementos que tienen una posición fija en dicho conjunto.
		\item ¿Cómo se construye una cadena en Python?\\R: Para construirlo, basta con poner los elementos en el orden deseado, separados por comas, y encerrarlos dentro de corchetes. ”[Esto, es, una, lista, de, 7, elementos]”. Puede asignársele un nombre y guardarla como si se tratase de una variable cualquiera.
		\item ¿Para qué sirve el método append() en las listas?\\R: El método .append(”valor”) agrega un elemento a la cola de la lista.
		\item ¿Para qué sirve el método .insert(posicion, valor)?\\R: Agrega en la posición deseada de la lista un valor dado, desplazando a la derecha el valor que ocupaba esa posición
		\item ¿Para qué sirve el método pop() y pop(posición)?\\R: Ese método quita, en el primer caso, de la lista el último elemento o, en el segundo caso, el elemento cuya posición se determina.
		\item ¿Para qué sirve el método .extend(lista)?\\R: Este método permite que a una lista dada se le puedan agregar los elementos de otra lista, que quedarían a la cola de la primera.
		\item ¿Para qué sirve la función len(lista)?\\R: Para conocer cuántos elementos tiene una lista. También funciona con cadenas.
		\item ¿Cómo se determinan las posiciones de los elementos de una lista?\\R: El primer elemento será el elmento 0, y el último será el elemento len(lista)-1, lo que indica que todos los elementos se enumerarán del 0 al len(lista)-1.
		\item ¿Cómo se puede conocer el elemento de una lista que está en una posición determinada?\\R: Sólo se requiere escribir lista[posición del elemento], con lo que el sistema arrojará el elemento deseado. En el caso de sublistas, que son listas dentro de una lista, se requiere escribir lista[posicion de sublista][posición del elemento].
		\item ¿Para qué sirve el comando for?\\R: Para permitir que una intrucción se repita a lo largo de cada elemento de una lista.
		\item ¿Para qué sirve el comando range(n)?\\R: Simplemente crea una lista cuyos elementos ordenados van del 0 al n-1
	\end{enumerate}
\end{document}