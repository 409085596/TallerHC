\documentclass[letterpaper, 12pt, oneside]{article} %Para dar formato al documento
\usepackage{amsmath}
\usepackage{graphicx}
\usepackage{xcolor}
\usepackage[utf8]{inputenc}

\title{\Huge Bitácora 14 del Taller de Herramientas Computacionales}
\author{Elías Jiménez Cruz, 409085596}
\date{25/01/2019}

\begin{document}
	\maketitle
	\paragraph{En la penúltima clase, se checaron dudas sobre la tarea del día anterior, que consistió en realizar los ejercicios de la sección de evaluación del libro Python fácil. Eran ocho ejercicios, de los cuales el que presentó más dudas fue el del laberinto. Consiste en lo siguiente: "Diseñe  y  programe  un  algoritmo  recursivo  que  encuentre  la  salida de  un laberinto, teniendo en cuenta que el laberinto se toma como entrada y que  es  una  matriz  de  valores  True,  False,  (x,y),  (a,b),  donde  True indica uno bstáculo;  False,  una  celda  por  la  que  se  puede  caminar; (x,y),  el  punto donde comienza a buscarse la salida y (a,b), la salida del laberinto.". Entonces, se checaron varias posibles soluciones, yo en particular realicé el algoritmo de la siguiente manera:}
	\begin{verbatim}
	{# -*- coding: utf-8 -*-
	'''Diseñe  y  programe  un  algoritmo  recursivo  que  encuentre  la  salida
	de  un laberinto, teniendo en cuenta que el laberinto se toma como entrada y
	que  es  una  matriz  de  valores  True,  False,  (x,y),  (a,b),  donde  True
	indica uno bstáculo;  False,  una  celda  por  la  que  se  puede  caminar;
	(x,y),  el  punto donde comienza a buscarse la salida y (a,b), la salida del
	laberinto.'''
	
	def laberinto(matriz, e):
	try:
	if matriz[e[0]-1][e[1]] == False:
	matriz[e[0]][e[1]] = True
	return(laberinto(matriz, (e[0]-1,e[1])))
	except:
	pass
	try:
	if matriz[e[0]][e[1]+1] == False:
	matriz[e[0]][e[1]] = True
	return(laberinto(matriz, (e[0],e[1]+1)))
	except:
	pass
	try:
	if matriz[e[0]+1][e[1]] == False:
	matriz[e[0]][e[1]] = True
	return(laberinto(matriz, (e[0]+1,e[1])))
	except:
	pass
	try:
	if matriz[e[0]][e[1]-1] == False:
	matriz[e[0]][e[1]] = True
	return(laberinto(matriz, (e[0],e[1]-1)))
	except:
	pass
	return("La salida es: "+str(e))}
	\end{verbatim}
	\paragraph{Esta es el archivo de implementación, el cual explica en qué consiste y como debe introducir los datos el usuario5:}
	\begin{verbatim}
	{# -*- coding: utf-8 -*-
	from Ejercicio05 import laberinto
	
	interruptor = True
	while interruptor == True:
	matriz= input('''
	Vamos a avergiuar cuál es la salida de un laberinto formado por una matriz.
	Para ello, dame primero una matriz boolenana, sin espacios entre los elementos,
	cuyas filas y columnas formarán las celdas por las que pasará el camino. Si el
	valor de la celda es True, se le considerará un bloque por el que no se
	podrá pasar, si es False, se podrá pasar por ahí. Así, las celdas que tengan
	el valor false deberán formar un camino entre las celdas True, pero sin que
	alguna celda False toque lateralmente a más de otra celda False.
	Así mismo, una vez ingresado el laberinto, presiona enter e ingresa en
	forma de tupla las coordenadas de filas y columnas donde se ubicará la entrada
	del laberinto, que será la celda donde se empezará a recorrer el camino hasta el
	final. Para dar mayor realismo al ejercicio, procura que la celda de entrada y
	la de salida sean adyacentes a los bordes de la matriz:
	''')
	e=input()
	print laberinto(matriz,e)
	interruptor = bool(raw_input("Si desea repetir el proceso ingrese cualquier 
	caracter, \nde lo contrario presione enter: "))
	}
	\end{verbatim}
\end{document}