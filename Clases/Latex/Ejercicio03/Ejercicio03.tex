\documentclass{book}
%\usepackage[spanish]{babel}
\usepackage[utf8]{inputenc}
%\usepackage{biblatex}
\usepackage{hyperref}

\title{Taller de Herramientas Computacionales}
\author{Elías Jiménez Cruz}
\date{17/Enero/2019}

\begin{document}
	\maketitle
	%Aquí inicia el índice del contenido del texto.
	\tableofcontents
	\section*{Introducción} Este libro es para fortalecer el conocimiento de la materia de Taller de Herramientas Computacionales.
	\url{www.google.com}
	\hyperref[Google]{www.google.com}
	
	%Aquí inician los capítulos del libro.
	\chapter{Uso básico de Linux}
	\section{Distribuciones de Linux}
	\section{Comandos}
	\chapter{Introducción a LateX}
	\chapter{Introducción a Python}
	\begin{verbatim}
	#!/usr/bin/python2.7
	# -*- coding: utf-8 -*-
	
	'''Elías Jiménez Cruz
	409085596
	Taller de herramientas computacionales
	{aquí va una descripción del programa y lo que hace}'''
	
	x= 10.5
	y= 1.0/3
	z= 15.3
	#x,y,z= 10.5,1.0/3,15.3
	H= '''El punto en R3 es (x,y,z) = (%.2f,%g,%G)''' %(x,y,z)
	print H
	
	G= '''El punto en R3 es (x,y,z) = ({laX:.2f},{laY:g},{laZ:G})''' .format(laX=x,laY=y,laZ=z)
	print G
	
	#import math as m
	#from math import sqrt
	#from math import sqrt as s
	
	from math import *
	x = input("Dame el número del cual quieres conocer su raíz: ")
	print "La raíz cuadrada de %.2f es %f" %(x,sqrt(x))
	\end{verbatim}
	\input{prueba.py}
	\section*{Orientación a objetos}
	
	\begin{thebibliography}{9}
		%\bibitem{Computación}
		Autor blah!\\
		\textit{cualquier cosa}
		blah!
	\end{thebibliography}
\end{document}