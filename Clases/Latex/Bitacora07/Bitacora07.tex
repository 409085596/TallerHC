\documentclass[letterpaper, 12pt, oneside]{article} %Para dar formato al documento
\usepackage{amsmath}
\usepackage{graphicx}
\usepackage{xcolor}
\usepackage[utf8]{inputenc}

\title{\Huge Bitácora 7 del Taller de Herramientas Computacionales}
\author{Elías Jiménez Cruz, 409085596}
\date{15/01/2019}

\begin{document}
	\maketitle
	\paragraph{Comenzamos a ver el comando while, el cual permite que una instrucción se repita varias veces hasta que deje de cumplirse una condición determinada. Básicamente se escribe while(condición): y debajo se pone de forma indentada la instrucción. Este comando es demasiado útil si se sabe utilizar. Se pueden realizar sumas de una sucesión de números, restas, e incluso se pueden poner con condicionales if, aumentando su poder. De esta forma, hicimos el ejercicio Ulam.py, en vistas de practicar. Usualmente se utiliza un contador en estos comandos, el cual lleva la cuenta de las veces que se repite la instrucción, y al llegar a un número determinado de veces hace que se detenga el ciclo. No obstante, la condición no necesariamente tiene que ser de naturaleza númerica, puede ser un valor booleano, es decir, de verdadero o falso, el cual es determinado desde dentro del ciclo mismo. Esto permite que las posibilidades se extiendan, y sea diferente a otras opciones de ciclos, los cuales se verán más adelante.}
\end{document}